\subsection{Rethinking IoT Service Architecture}

Among the IoT framework services shown in Figure~\ref{fig:service-arch}, two play a more fundamental role:
\begin{enumerate}
\item \emph{Trust management}, which mandates how the IoT system entities (users, devices, and applications) can authenticate each other;
\item \emph{Rendezvous}, which provides a means for different entities to reach each other over the local and global networks.
\end{enumerate}

Independent from the lower layer addressing scheme, trust management and rendezvous at the application layer are built on the concept of \emph{named entities}.
The application-layer names are either specified by the users or auto-generated by the devices and applications.
Those names serve as the \emph{identities} of the entities in an IoT system and allow those entities to refer to each other and interoperate.
To enable authentication, the identity names are usually associated with some form of credentials such as user passwords and public key certificates.
Therefore application-layer trust policies can be composed in terms of names rather than low-level security material.
Named entities also provide the basis for rendezvous: by exchanging names, entities can find out about other entities on the network. 

All other IoT services can be bootstrapped from these two core services.
For example, device management is based on the mutual trust and direct communication (via rendezvous) between devices and managing services;
resource discovery is a natural extension of rendezvous, enhanced by the ability to verify a resource's origin;
pub-sub messaging and external gateways both require the interplay of authentication and rendezvous functions.
Note that those services may also be interdependent on each other; together they form a framework layer on top of which IoT developers can create high-level applications and services.

%Today's cloud-centric IoT ecosystems typically provide most (if not all) of the framework-level and application-level services in Figure~\ref{fig:service-arch}.
A common argument for hosting IoT services and applications in the cloud is that the IoT system would become too complicated for ordinary users if they have to deal with the complexity of registering devices with some local controller, connecting devices with local and remote applications, and managing security credentials.
However, we argue that the usability problem can be addressed fundamentally with a hierarchical and human-friendly naming design because:
(1) human-friendly naming provides intuitive understanding of the trust relationship among devices and applications for non-expert users;
(2) hierarchical naming structure facilitates applications in expressing and exploring the organization of the IoT system, which further enables automated trust management and rendezvous tools.

Unfortunately, in TCP/IP network architecture the first-order names for devices and services are IP addresses,%
\footnote{For example, in cloud computing it is common practice to assign one or more \textit{virtual IPs (VIPs)} to identify cloud services. VIPs are often different from the IP addresses of the physical machines that host the services.}
accompanied by per-device or per-service public keys for (D)TLS authentication.
Human-readable names, such as URLs, are application-layer aliases that have to be resolved when the applications access data or services via the network.
Numeric names are straightforward for machines to operate on, but contain no semantic meaning that can be leveraged to make trust decisions, support rendezvous, or provide intuitive understanding for human users and application developers.
%The numeric names, while easy to handle by the servers in the cloud, are unintuitive for human users and application developers.
The NDN architecture offers an elegant solution to the naming problem at the network layer. It allows the IoT services to be described locally and in a decentralized way without sacrificing security, functionality, flexibility, and usability (described in the next subsection).

\subsection{Achieving Local IoT Functions with NDN}

The NDN architecture bootstraps proximal IoT communication by naming the entities in the context of the local IoT network.
Instead of obtaining their identities from cloud service providers, the entities in the IoT network create local identities associated with asymmetric cryptographic keys that are certified by a local trust anchor.
The identity certificates are all published as Data packets in the local NDN network under the identity namespace.
The trust anchor is typically a root key created by the manager of the IoT system and stored securely on a local authentication server such as a control hub or a TPM-equipped smartphone.
With the support from appropriate tools, the key and certificate management tasks can be made user-friendly enough for non-experts.

Note that the entities may also have other identities for communication outside the IoT network.
For example, the devices may have manufacture-issued identities that are used for signing the device state reports or retrieving software/firmware updates;
the users may also have public identities (e.g., OpenIDs) that can be used for initial authentication when new users are added to the system.
The practice of using different identities for different purposes is aligned with the \emph{principle of least privilege}.

After the local identities are created, the NDN-IoT architecture can leverage two powerful tools to provide the two fundamental services: using \emph{schematized trust}~\cite{trust-schema} for local trust management, and \emph{distributed sync}~\cite{chronosync} for local rendezvous.

\subsubsection{Trust management}
The trust policies for the IoT system can be expressed by the \emph{trust schema}, which specifies the relations between data name and signing key name using a domain-specific language designed for pattern matching on NDN names.
The NDN software platform provides tools that automatically sign and verify the Data packets according to the pre-defined trust schema, which can be integrated into the applications through client libraries.

\subsubsection{Rendezvous}
NDN Sync protocols such as \emph{ChronoSync} allow multiple devices to synchronize the namespace of the shared dataset without relying on central servers.
This mechanism provides a convenient rendezvous solution where the application prefixes and device identities are published under a well-known namespace and synchronized across multiple devices in the IoT network.
For large-scale IoT systems, multiple sync groups can be created under separate namespaces to isolate independent subsystems.

Through support for decentralized trust management and rendezvous, the NDN-IoT architecture enables cloud-independence while providing essential IoT services.
Applications can still benefit from the cloud whenever they need, such as storing large amount of data, performing complex data analysis jobs, or accessing external services like search and voice recognition.
In fact, NDN simplifies the integration with the external networks by using the universal Interest-Data exchange primitive for data communication.
The cloud services become an optional component, rather than the central piece in the architecture.

The automated local trust management and rendezvous also improve user experience because the only manual step of configuration during device setup is to register the device with the local trust anchor and obtain certified local identity for the device.
Once the trust is established, the IoT devices and applications can communicate with each other, exchange useful information, or discover new devices and applications without human intervention.